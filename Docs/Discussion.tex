\chapter{Discussion} 
This chapter will lift our thoughts on our result, our theoretical not yet implemented optimizations and
what future work is possible from this point in time.
	
	\section{Results}  \label{discussion}
		
		\subsection{Software Shader}
		
			The implemented optimizations improved performance more than anticipated.
			This shows that there is more performance to be found in a conventional
			GPU than a straight-forward implementation of the algorithm. 

			There are great limitations in what could be done with the shader because
			of the language that we chose to implement it in, GLSL. It offers nothing 
			more than a way to write a program which will be run on a per-pixel basis
			and put it on the screen. Because of this, CPU-GPU cooperation is completely
			impossible and that comes with consequences.

			Modern graphics engines not only do a lot of work on the
			CPU but it has granular control over the GPU, whereas we have no control
			at all. Things such as object positioning, frustum culling, etc. should be 
			done by the CPU, not the GPU. This kind of implementation can 
			not be done in GLSL and therefore a high-performance sphere tracer should be done not in 
			GLSL but in a lower-level language.

			Although for our purposes, to build a sphere tracer where we quickly could implement
			optimizations, develop new features, study the algorithm, GLSL was a
			perfect fit. 

		
		\subsection{GPU} 
		
			The project originally intended to create a simple single-core GPU
			that could fully render a scene given enough time, and then add on
			components as time allowed. As happens with projects like these,
			every design decision one makes clarifies what further needs to be
			done and reveals previously unseen flaws in the design that needs
			to be resolved. In the end we implemented the core and enabled
			multiple cores running in parallel utilizing a global queue, with
			some caveats. While the implementation works and can be simulated
			for up to 128 cores we have only successfully serialized 4 cores. 
			
			If we were to continue the work in a future project we would
			rewrite the basic core in a number of ways, pushing it out to FPGA
			to verify our design choices and try to be more empiric before
			extending further with more modules and ideas. 

	\section{Optimizations} \label{optimization}

		\subsection{Ray grouping}
		
			Ray grouping is an optimization that lowers the number of
			computations the GPU have to perform. Also by lowering the number of
			computations it increases the speed of the GPU.
			
			It works like the name implicates by grouping adjacent rays
			together. If a specific pixel N is to be rendered we group the
			adjacent pixels into a "combined pixel". Then we march along the
			ray of N. If any of the other pixels are not inside the minimum
			distance spheres (MDS) the combined pixel is split up into two
			smaller groups. Each of the groups then repeat the first step again
			individually along their new center pixel's ray. This is repeated a
			number of times depending on the scene.  The closer to a target it
			gets, more subgroups will be created due to the MDS's volumes will
			decrease.
			
			This optimization was implemented and we achieved an increase in
			performance as excepted. The optimization decreases the number of
			computations made by the formula ? below.  Sum of $\sum_1^s(1/N_s)$
			where N = number of pixels in the group and s = number of groups

			%#picture of ray grouping

		\subsection{Bounding spheres}
			
			Although an increase in performance could be seen using this method,
			it can also be used in ways that decreases the performance. By
			setting up a too large boundnig sphere or my setting it up with few
			objects or simply objects that are far apart.

			To use this optimiztion efficiently the objects inside a bounding 
			sphere should never come far apart and the sphere should be exactly 
			the right size.

		\subsection{Orthogonal culling}

			The performance gain from this optimization exeeded our 
			expectations. This could probably be further improved by using
			another way to create the sphere tracer than GLSL. 

			However great the performance gain the way this is implemented
			is still far from optional, currently it performs orthogonal 
			projection on all the objects in the scene per pixel. Instead,
			calculations should be done on a per-object basis and calculate
			which rays a object intersects, this way fewer calculations would
			have to be performed and optionally they would be performed on the
			CPU.

			This method could work together with the Bounding Sphere 
			optimization. Instead of projecting single objects onto the ray,
			bounding spheres could be projected, decreasing the number of 
			orthogonal projections that has to be made. This too was tested 
			and an increase in performance was observed.

	\section{Future work}

		\subsection{GPU-CPU cooperation}
		
			One thing that could be done to improve performance is to let the
			processor handle object related math, such as object and camera 
			transformation. Currently if objects in the scene are supposed to 
			move without doing it manually, they have to be moved using 
			mathematical functions. These functions are then evaluated for 
			each march step, for each pixel, this is essentially wasted 
			computing power.


		\subsection{Plane-ray intersection}

			Infinite planes currently require a lot of computing power to render. When
			the camera is oriented so that the field of view is along an infinite
			plane some rays will travel parallell to the plane, these rays will march 
			equally long steps until their max number of steps has been taken or
			the max range has been reached, without hitting anything. Another
			drawback of infinite planes is that the sphere tracing algorithm causes
			the horizon of planes to disappear because of the max range and max 
			steps of a ray. Another way to render infinite planes might be to 
			calculate if the ray will intersect the plane, and if so where it 
			will intersect. This would eliminate the issue of not being able 
			to render the horizon and might also increase performance.

			One potential problem is that because no distance field evaluations 
			are performed, it might be hard to perform operations such as CSG or
			mathematical deformations on planes rendered this way.

		\subsection{Overstepping}

			Bounding spheres technique is somewhat similar to the normal sphere
			tracing. The difference is that when you march you march to the
			edge of the MDS and then you march to the next MDS edge. With
			overstepping technique you march a small distance further
			outside the MDS edge. You then compare the original MDS with the
			new MDS if these two spheres overlap in any way we can march that
			little bit further. By marching that little bit further, decrease
			in the number of times marched is achieved. Giving an increase in
			performance. 
			
