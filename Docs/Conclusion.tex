\chapter{Conclusions}

	The goal of the project was to design and create a GPU built for the
	Sphere Tracing algoritm and try to improve their performance. A
	working GPU was written together with a Sphere Tracing shader
	programmed, separately, running on the GPU. We also implemented the
	GPU on an FPGA and writing our output to the screen on the FPGA. The
	GPU executes as intended and, if programmed correctly, can Sphere
	Trace complete scenes.
	
	A similar reference shader was also written in GLSL. On the GLSL
	shader we tried two optimizations, Bounding Sphere and Orthogonal
	Culling. Bounding spheres decreases the number of calculations made
	in each march by enclosing several objects into one bounding sphere.
	Orthogonal Culling is used to tell whether a object is in front of
	the ray or not, decreasing the number of object needed to calculate
	the distance to. Both improved the performance but only by one order
	of magnitude. They can still be further improved by expanding the
	shader to an engine where parts of the computation would be relieved
	by the CPU.
	
	For the hardware optimization, square roots was a bottle neck both in
	speed and area. With that said, we explored ways of speeding up the
	square root. We found a number of fast solutions that are all bounded
	in the relative error.
	
	It has been a challenging project for all of the group members due to
	the scope of the project and the lack of experience in both hardware
	design and graphics overall. This lead to a lot of reading up on a lot
	of different subjects in tandem???. Not only knowledge about the parts in
	the project but also the parts around the project. The importance of
	setting up sectoral targets and following them as close as possible.
	In our case getting a first working version of the GPU to be able to
	continue working from there. Also keeping all project members busy
	simultaneously, much like you it is sought for in designing hardware
	and all of its components.

