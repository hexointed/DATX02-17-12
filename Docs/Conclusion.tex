\chapter{Conclusions}

	The goal of the project was to design and create a GPU for the Sphere Tracing
	algorithm and try to improve its performance. A working GPU was designed
	together with two shaders, one written in our own assembly language running
	on the GPU and one written in GLSL running on a traditional consumer graphics
	card. We also simulated the GPU, rendering to a small virtual screen. The GPU
	executes as intended and, if programmed correctly, can Sphere Trace complete
	scenes.
	
	For the GLSL shader we describe and tried three optimizations, Bounding
	Spheres, Orthogonal Culling and Ray Grouping. Bounding Spheres decrease the
	number of calculations made in each march by enclosing several objects into
	one bounding sphere. Orthogonal Culling is used to tell whether a object is
	in front of the ray or not, decreasing the number of objects needed to
	calculate the signed distance function. Ray Grouping decreases the number of
	of steps needed to march by grouping rays together into blocks. All of them
	improved performance. We believe they can be improved further by offloading
	parts of the computation to the CPU.
	
	For the hardware optimization, calculating square roots was considered
	important and methods for calculating them efficiently where investigated. We
	found a number of fast approximations that are bounded in their relative
	error. We also looked at possible optimizations for an exact integer square
	root algorithm.
