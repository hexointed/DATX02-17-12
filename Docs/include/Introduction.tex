% CREATED BY DAVID FRISK, 2016
\chapter{Introduction} 

The algorithm in question that we designed a custom GPU for is called sphere
tracing.\cite{Hart1996} So called since it uses spheres to incrementally
advance a ray\footnotemark in 3D space. The method of advancing rays
incrementally is called ray marching and is a particular subset of ray
tracing.\cite{Whitted1980} Ray tracing, then, is a way of wholly or partially
rendering the world through rays, cast from the eye of the observer into the
scene.  Sphere tracing has been around since at least as early as the late
eighties and ray tracing as early as the sixties.\cite{Hart1989,Appel1968}
Since ray tracing traditionally has been a more computation-intensive method
compared to scanlining\cite{Wylie1967} an as such it has generally seen more
use in movie production rather than in real-time applications.\cite{ref_needed?} 

%TODO(bjorn): Nån som är bätre än mig på detta får gärna skriva om denna delen
              %nedanför. låter inte proffsigt alls.

The advent of the programmable shader brought back the discussion of real time
ray marching to the forefront. which is where we caught on to it.
\cite{JamieWong2016} It still favours poorly compared to other techniques wich
natrually begged the question if it would be able to compete in real time if
only the hardware for it was there.

\footnotetext{vector that emerges and bounces from a view-point} 

\section{Sphere tracing}

At the base of the algorithm are the signed distance functions
$SDF:\mathbb{R}^{3}\mapsto\mathbb{R}$. The "distance" is the distance between a
point and the closest point on the implicit surface $SDF^{-1}(0)$. The "signed"
part refers to the distance being negated when measured inside of the surface.
If we define a $ray(s) = \vec{d} \cdot s + \vec{o}$ where $\vec{d}$ is the
normalized direction of the ray and $\vec{o}$ the origin, then $SDF\circ ray(s)
  = 0$ means that the ray intersects a surface at distance $s$. Finding the
  surface can then be done by iterating point by point from the origin along
  the ray.  $p_{i+1} = p_i + \vec{d}\cdot SDF(p_i)$ which is repeated until
  $SFD(p) = 0$.  $SDF(p_i)$ is the furthest we can march the ray while still be
  sure we don't overshoot any potential surfaces. The direction of the closest
  surface point is never known thus $SDF(p_i)$ can be interpreted as a sphere
  bound, giving the algorithm its namesake. This ray marching is then performed
  for each pixel of the screen reversely simulating the light from a scene
  entering the lens of the onlooker.

\subsection{Reflections and refractions}

Once a point on a surface for a given pixel has been located new multiple rays
can then be further marched to determine reflections, towards the scene's
sources of light determening light and shadows, or through the object with an
angle, simulating refractions. A lot of these depend on the surface normal
which can be calculated by normalizing the aproximate gradient of $SDF$. 

\subsection{Textures}



\section{Hardware implementation}


