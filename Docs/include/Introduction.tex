% CREATED BY DAVID FRISK, 2016
\chapter{Introduction} 

The algorithm in question that we designed a custom GPU for is called sphere
tracing.\cite{Hart1996} So called since it uses spheres to incrementally
advance a ray\footnotemark in 3D space. The method of advancing rays
incrementally is called ray marching and is a particular subset of ray
tracing.\cite{Whitted1980} Ray tracing, then, is a way of wholly or partially
rendering the world through rays, cast from the eye of the observer into the
scene.  Sphere tracing has been around since at least as early as the late
eighties and ray tracing as early as the sixties.\cite{Hart1989,Appel1968}
Since ray tracing traditionally has been a more computation-intensive method
compared to scanlining\cite{Wylie1967} an as such it has generally seen more
use in movie production rather than in real-time applications.\cite{ref_needed?} 

%TODO(bjorn): Nån som är bätre än mig på detta får gärna skriva om denna delen
              %nedanför. låter inte proffsigt alls.

The advent of the programmable shader brought back the discussion of real time
ray marching to the forefront. which is where we caught on to it.
\cite{JamieWong2016} It still favours poorly compared to other techniques wich
natrually begged the question if it would be able to compete in real time if
only the hardware for it was there.

\footnotetext{vector that emerges and bounces from a view-point} 

\section{Sphere tracing}

At its most simple form 

\section{Hardware implementation}
