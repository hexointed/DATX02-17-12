% CREATED BY DAVID FRISK, 2016
\chapter{Introduction} 

The primary goal of the project is a ray-marching graphics card
proof-of-concept that hopefully will look promising enough to warrant further
work in making a ray-marching based graphics card. If the results show that
there is more performance potential in ray-marching than in the current
industry’s approach, then that could become a link from the theory of
ray-marching to the practice of hardware-accelerated ray-marching. This would
be useful in order to more accurately estimate what kind of performance you
could expect if the current industry focused on ray-marching instead of polygon
based graphics.

The aim of this project is to develop the fundamental building blocks for a real-
time 3D graphics processing unit (GPU) that is optimized to run a rendering 
algorithm which is different to the Polygon-based rendering that is almost 
exclusively used in today's 3D graphics cards. During the early 90’s many types of 
rendering algorithms were developed and used, since they all ran in software on the 
main computer processor. Better performance (better graphics) was a key selling 
point among competing computer systems, the computer industry moved towards 
dedicated hardware based rendering systems. These systems were almost exclusively 
using polygon-based rendering. Today, 3D graphics cards have become more 
programmable but are still based on the old paradigm of polygon rendering. The last 
5 years have seen a small resurgence in the use of a sphere tracing rendering 
algorithm known as Ray- Marching. This has a whole new set of pros and cons when 
compared to the polygon renderers.  However, while the algorithm seems to have a lot 
of potential for future 3D graphics, that potential might be unreachable on current 
GPUs due to the underlying architecture being designed with other things in mind. 
This project will examine what could be achieved if graphics cards were optimized 
for ray-marching instead of drawing polygons.  



\section{Section levels}
The following table presents an overview of the section levels that are used in this 
document. The number of levels that are numbered and included in the table of contents is 
set in the settings file \texttt{Settings.tex}. 
The levels are shown in Section \ref{Section_ref}.

\begin{table}[H]
\centering
\begin{tabular}{ll} \hline\hline
Name & Command\\ \hline
Chapter & \textbackslash\texttt{chapter\{\emph{Chapter name}\}}\\
Section & \textbackslash\texttt{section\{\emph{Section name}\}}\\
Subsection & \textbackslash\texttt{subsection\{\emph{Subsection name}\}}\\
Subsubsection & \textbackslash\texttt{subsubsection\{\emph{Subsubsection name}\}}\\
Paragraph & \textbackslash\texttt{paragraph\{\emph{Paragraph name}\}}\\
Subparagraph & \textbackslash\texttt{paragraph\{\emph{Subparagraph name}\}}\\ \hline\hline
\end{tabular}
\end{table}

\section{Section} \label{Section_ref}
\subsection{Subsection}
\subsubsection{Subsubsection}
\paragraph{Paragraph}
\subparagraph{Subparagraph}

