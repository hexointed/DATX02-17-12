% CREATED BY DAVID FRISK, 2016
\varHeadline\\
\varSubtitle\\
\varNames\\
\varDepartment\\
Chalmers University of Technology \setlength{\parskip}{0.5cm}

\thispagestyle{plain}			% Supress header 
\setlength{\parskip}{0pt plus 1.0pt}
\section*{Abstract}
The aim of this project is to develop the fundamental building blocks for a real-
time 3D graphics processing unit (GPU) that is optimized to run a rendering 
algorithm which is different to the Polygon-based rendering that is almost 
exclusively used in today's 3D graphics cards. During the early 90’s many types of 
rendering algorithms were developed and used, since they all ran in software on the 
main computer processor. Better performance (better graphics) was a key selling 
point among competing computer systems, the computer industry moved towards 
dedicated hardware based rendering systems. These systems were almost exclusively 
using polygon-based rendering. Today, 3D graphics cards have become more 
programmable but are still based on the old paradigm of polygon rendering. The last 
5 years have seen a small resurgence in the use of a sphere tracing rendering 
algorithm known as Ray- Marching. This has a whole new set of pros and cons when 
compared to the polygon renderers.  However, while the algorithm seems to have a lot 
of potential for future 3D graphics, that potential might be unreachable on current 
GPUs due to the underlying architecture being designed with other things in mind. 
This project will examine what could be achieved if graphics cards were optimized 
for ray-marching instead of drawing polygons.  

This report documents the results of us taking the ray-marching algorithm
\footnote{for 3D real-time rendering, as opposed to ray-tracing etc.} and doing
a first-pass design of a GPU with some optimizations applied. We begin with an
in-depth description of the algorithm and the simplest instance of the GPU. 


% KEYWORDS (MAXIMUM 10 WORDS)
\vfill
Keywords: lorem, ipsum, dolor, sit, amet, consectetur, adipisicing, elit, sed, do.

\newpage				% Create empty back of side
\thispagestyle{empty}
\mbox{}
