% CREATED BY DAVID FRISK, 2016

% IMPORT SETTINGS
\documentclass[12pt,a4paper,twoside,openright]{report}
% CREATED BY DAVID FRISK, 2016

% NOTE(Bjorn): Added packages.
\usepackage{verbatim}                   % For block comments
\usepackage{verbatimbox}
\usepackage{cite}
% BASIC SETTINGS
\usepackage{moreverb}                   % List settings
\usepackage{textcomp}                   % Fonts, symbols etc.
\usepackage{lmodern}                    % Latin modern font
\usepackage{helvet}                     % Enables font switching
\usepackage[T1]{fontenc}                % Output settings
\usepackage[english]{babel}             % Language settings
\usepackage[utf8]{inputenc}             % Input settings
\usepackage{amsmath}                    % Mathematical expressions
\usepackage{amssymb}                    % Mathematical symbols
\usepackage{graphicx}                   % Figures
\usepackage{wrapfig}					% Figures with text wrapping
\usepackage{subfig}                     % Enables subfigures
\numberwithin{equation}{chapter}        % Numbering order for equations
\numberwithin{figure}{chapter}          % Numbering order for figures
\numberwithin{table}{chapter}           % Numbering order for tables
\usepackage{listings}                   % Enables source code listings
\usepackage{chemfig}                    % Chemical structures
\usepackage[top=3cm, bottom=3cm,
            inner=3cm, outer=3cm]{geometry}     % Page margin lengths
\usepackage{eso-pic}                    % Create cover page background
\newcommand{\backgroundpic}[3]{
        \put(#1,#2){
        \parbox[b][\paperheight]{\paperwidth}{
        \centering
        \includegraphics[width=\paperwidth,height=\paperheight,
                         keepaspectratio]{#3}}}}
\usepackage{float}                      % Enables object position enforcement using[H]
\usepackage{parskip}                    % Enables vertical spaces correctly 



% Caption settings (aligned left with bold name)
\usepackage[labelfont=bf, textfont=normal,
                        justification=justified,
                        singlelinecheck=false]{caption}                 

                        
% Activate clickable links in table of contents         
\usepackage{hyperref}                                                           
\hypersetup{colorlinks, citecolor=black,
            filecolor=black, linkcolor=black,
            urlcolor=black}


% Define the number of section levels to be included in the t.o.c. 
% and numbered (3 is default)   
\setcounter{tocdepth}{5}                                                        
\setcounter{secnumdepth}{3}     


% Chapter title settings
\usepackage{titlesec}           
\titleformat{\chapter}[display]
  {\Huge\bfseries\filcenter}
  {{\fontsize{50pt}{1em}\vspace{-4.2ex}\selectfont \textnormal{\thechapter}}}{1ex}{}[]


% Header and footer settings (Select TWOSIDE or ONESIDE layout below)
\usepackage{fancyhdr}                                                           
\pagestyle{fancy}  
\renewcommand{\chaptermark}[1]{\markboth{\thechapter.\space#1}{}} 
\newcommand{\clash}{C$\lambda$aSH }


% Select one-sided (1) or two-sided (2) page numbering
\def\layout{2}  % Choose 1 for one-sided or 2 for two-sided layout
% Conditional expression based on the layout choice
\ifnum\layout=2 % Two-sided
    \fancyhf{}                                                                  
        \fancyhead[LE,RO]{\nouppercase{ \leftmark}}
        \fancyfoot[LE,RO]{\thepage}
        \fancypagestyle{plain}{                 % Redefine the plain page style
        \fancyhf{}
        \renewcommand{\headrulewidth}{0pt}              
        \fancyfoot[LE,RO]{\thepage}}    
\else                   % One-sided     
        \fancyhf{}                                      
        \fancyhead[C]{\nouppercase{ \leftmark}}
        \fancyfoot[C]{\thepage}
\fi


% Enable To-do notes
\usepackage[textsize=tiny]{todonotes} % Include the option "disable" to hide all notes
\setlength{\marginparwidth}{2.5cm} 


% Supress warning from Texmaker about headheight
\setlength{\headheight}{15pt}           


\begin{document} 

% COVER PAGE, TITLE PAGE AND IMPRINT PAGE
% Roman numbering (starting with i (one)) until first main chapter
\pagenumbering{roman}

%%%%%%%%%%%%%%%%%%%%%%%%%%%%%%%%%%%%%%%%%%%%%%%%%%%%%%%%%%
% NOTE(Bjorn):Global Variables for the frontmatter section.
%%%%%%%%%%%%%%%%%%%%%%%%%%%%%%%%%%%%%%%%%%%%%%%%%%%%%%%%%%
% An Informative Headline describing\\ the Content of the Report
\newcommand{\varHeadline}{Opponering av DATX02-17-10}
% A Subtitle that can be Very Much Longer if Necessary
\newcommand{\varSubtitle}{Av DATX02-17-12}
% Department of Some Subject or Technology
\newcommand{\varDepartment}{Department of Computer Science and Engineering}
% NAME FAMILYNAME
\newcommand{\varNames}{André Perzon, Björn Strömberg, Chi Thong Luong,  \\
Elias Forsberg, Jesper Åberg, Jon Johnsson}

\thispagestyle{empty}
\begin{center}
	\textsc{\large Bachelor of Science Thesis}\\[4cm]
	\textbf{\Large\varHeadline} \\[1cm]
	{\large \varSubtitle}\\[1cm]
	{\large \varNames}
	
	\vfill	
	
	\varDepartment \\
	\textsc{Chalmers University of Technology} \\
	Gothenburg, Sweden, July 2017 \\
\end{center}

\setlength{\parindent}{1cm}                         


% Opponering

\section*{Inledning}

Detta är en opponering på rapporten ``Virtual Generation of Lidar Data for
Autonomous Vehicles‘‘. I denna opponeringen ger vi konstruktiv kritik och
förslag angående innehållet och strukturen på rapporten. Det finns en bilaga
som noterar alla funna detaljfel för att underlätta den kommande
kompletteringen. Vi börjar med att gå över alla kapitel för sig. 

\section*{Sammanfattning}

Sammanfattningen är koncis och representativ för projektet.

\section*{Introduction}

The beginning of the introduction gets the reader up to speed and explains why
the thesis is relevant. All references except [4] either says something
conflicting to the text or just not supports the text and as such needs some
more scrutinization. 1.1 express what you intend to do clearly but the second
paragraph belongs to the beginning of introduction. 1.2 state what the
overarching problem of the thesis is but does not further clarify other
challenges. 1.3 is fine but in the last paragraph (insert jon rant).

\section*{Technical Background}

Here the terms introduced in the previous chapter are further clarified as
expected. 2.2.1 and 2.2.2 could maybe be merged to one section and 2.2 could
arguably be called Unity instead of Game Engine since Unity is the sole focus
of the section. There could be a better picture in figure 2.1 to explain the
concept of collision and the subtext in figure 2.2 is redundant. 2.2.3 is clear
overall. 2.2.4 could be renamed to Pathfinding since that is the specific
subject being discussed. It is a bit more rough put the point gets across.

\section*{Methods}

This chapter's sections describes the methodology, requirement and the final
choice that was made in creating the components. It sounds a bit wide for a
"methods" section. 3.1 could become clearer by showing the math of the
ray-casting algorithm in detail. Also, from what we could find reading up on
Unity a bit, (elias rant on multithreading). 3.2 is fine overall but while the
thesis uses scene in general to refer to the part that is created by the user,
"world", "environment" and "space" is also sometimes used and it can get a bit
confusing. The first paragraph in 3.2.2 is just a reiteration of 3.1. In 3.2.3
you mention that there are alternative methods and that there where tests to
confirm decisions. Expand on this. 3.2.4 Is too small and seems to be out of
place. It could maybe be a separate section in this chapter. 3.1 is good in
general but 3.3.1 (elias rant). 3.4 was quite tough to parse and might be
helped by some pictures and relating the concepts discussed to your application
in paticular.

\section*{Results}


\section*{Discussion}

The beginning of 5.1 just mention further content and needs to be fleshed out
some more. It is still a bit vague in 5.1.1 as to how things have been
verified. This is maybe something that could be expanded upon in Methods
beforehand. The first paragraph in 5.1.2 is written as more of a result and
could instead be a reference to the results. The rest of the section could
maybe expand on whether there were any alternative ways to reduce the
performance that weren't explored. 5.2.1, 5.3 and 5.4 are reiterations of previous
points. In 5.3.2 you mention that JSON is not directly readable, but it is in a
text editor. You also conclude that the simulation depends on the hardware, but
this is always true to some extent. It would help if you were more specific,
maybe figure out some required minimum spec. 5.5 is an interesting read.

\section*{Conclusion}

The conclusion has some minor nitpicks mentioned in the appendix but gives
overall a good summation.

\section*{Slutsats}

\end{document}
