% CREATED BY DAVID FRISK, 2016

% IMPORT SETTINGS
\documentclass[12pt,a4paper,twoside,openright]{report}
% CREATED BY DAVID FRISK, 2016

% NOTE(Bjorn): Added packages.
\usepackage{verbatim}                   % For block comments
\usepackage{verbatimbox}
\usepackage{cite}
% BASIC SETTINGS
\usepackage{moreverb}                   % List settings
\usepackage{textcomp}                   % Fonts, symbols etc.
\usepackage{lmodern}                    % Latin modern font
\usepackage{helvet}                     % Enables font switching
\usepackage[T1]{fontenc}                % Output settings
\usepackage[english]{babel}             % Language settings
\usepackage[utf8]{inputenc}             % Input settings
\usepackage{amsmath}                    % Mathematical expressions
\usepackage{amssymb}                    % Mathematical symbols
\usepackage{graphicx}                   % Figures
\usepackage{wrapfig}					% Figures with text wrapping
\usepackage{subfig}                     % Enables subfigures
\numberwithin{equation}{chapter}        % Numbering order for equations
\numberwithin{figure}{chapter}          % Numbering order for figures
\numberwithin{table}{chapter}           % Numbering order for tables
\usepackage{listings}                   % Enables source code listings
\usepackage{chemfig}                    % Chemical structures
\usepackage[top=3cm, bottom=3cm,
            inner=3cm, outer=3cm]{geometry}     % Page margin lengths
\usepackage{eso-pic}                    % Create cover page background
\newcommand{\backgroundpic}[3]{
        \put(#1,#2){
        \parbox[b][\paperheight]{\paperwidth}{
        \centering
        \includegraphics[width=\paperwidth,height=\paperheight,
                         keepaspectratio]{#3}}}}
\usepackage{float}                      % Enables object position enforcement using[H]
\usepackage{parskip}                    % Enables vertical spaces correctly 



% Caption settings (aligned left with bold name)
\usepackage[labelfont=bf, textfont=normal,
                        justification=justified,
                        singlelinecheck=false]{caption}                 

                        
% Activate clickable links in table of contents         
\usepackage{hyperref}                                                           
\hypersetup{colorlinks, citecolor=black,
            filecolor=black, linkcolor=black,
            urlcolor=black}


% Define the number of section levels to be included in the t.o.c. 
% and numbered (3 is default)   
\setcounter{tocdepth}{5}                                                        
\setcounter{secnumdepth}{3}     


% Chapter title settings
\usepackage{titlesec}           
\titleformat{\chapter}[display]
  {\Huge\bfseries\filcenter}
  {{\fontsize{50pt}{1em}\vspace{-4.2ex}\selectfont \textnormal{\thechapter}}}{1ex}{}[]


% Header and footer settings (Select TWOSIDE or ONESIDE layout below)
\usepackage{fancyhdr}                                                           
\pagestyle{fancy}  
\renewcommand{\chaptermark}[1]{\markboth{\thechapter.\space#1}{}} 
\newcommand{\clash}{C$\lambda$aSH }


% Select one-sided (1) or two-sided (2) page numbering
\def\layout{2}  % Choose 1 for one-sided or 2 for two-sided layout
% Conditional expression based on the layout choice
\ifnum\layout=2 % Two-sided
    \fancyhf{}                                                                  
        \fancyhead[LE,RO]{\nouppercase{ \leftmark}}
        \fancyfoot[LE,RO]{\thepage}
        \fancypagestyle{plain}{                 % Redefine the plain page style
        \fancyhf{}
        \renewcommand{\headrulewidth}{0pt}              
        \fancyfoot[LE,RO]{\thepage}}    
\else                   % One-sided     
        \fancyhf{}                                      
        \fancyhead[C]{\nouppercase{ \leftmark}}
        \fancyfoot[C]{\thepage}
\fi


% Enable To-do notes
\usepackage[textsize=tiny]{todonotes} % Include the option "disable" to hide all notes
\setlength{\marginparwidth}{2.5cm} 


% Supress warning from Texmaker about headheight
\setlength{\headheight}{15pt}           


\begin{document} 

% COVER PAGE, TITLE PAGE AND IMPRINT PAGE
% Roman numbering (starting with i (one)) until first main chapter
\pagenumbering{roman}

%%%%%%%%%%%%%%%%%%%%%%%%%%%%%%%%%%%%%%%%%%%%%%%%%%%%%%%%%%
% NOTE(Bjorn):Global Variables for the frontmatter section.
%%%%%%%%%%%%%%%%%%%%%%%%%%%%%%%%%%%%%%%%%%%%%%%%%%%%%%%%%%
% An Informative Headline describing\\ the Content of the Report
\newcommand{\varHeadline}{Opponering av DATX02-17-10}
% A Subtitle that can be Very Much Longer if Necessary
\newcommand{\varSubtitle}{Av DATX02-17-12}
% Department of Some Subject or Technology
\newcommand{\varDepartment}{Department of Computer Science and Engineering}
% NAME FAMILYNAME
\newcommand{\varNames}{André Perzon, Björn Strömberg, Chi Thong Luong,  \\
Elias Forsberg, Jesper Åberg, Jon Johnsson}

\thispagestyle{empty}
\begin{center}
	\textsc{\large Bachelor of Science Thesis}\\[4cm]
	\textbf{\Large\varHeadline} \\[1cm]
	{\large \varSubtitle}\\[1cm]
	{\large \varNames}
	
	\vfill	
	
	\varDepartment \\
	\textsc{Chalmers University of Technology} \\
	Gothenburg, Sweden, July 2017 \\
\end{center}

\setlength{\parindent}{1cm}                         


% Opponering

\section*{Introduction}
	
	This is an opposition report on ``Virtual Generation of Lidar Data for
	Autonomous Vehicles‘‘, where we provide feedback and suggestions for the
	content and structure of the report. Our comments are ordered by chapter in
	the original report. An appendix with more specific comments, including
	typos and gramatical errors has been included to aid the authors.

\section*{Abstract}

	The abstract is concise and representative for the project.

\section*{Introduction}

	The beginning of the introduction gets the reader up to speed and explains
	why the thesis is relevant. All references except [4] either say something
	conflicting to the text or does just not support the text, requiring more 
	scrutiny. 1.1 - express what you intend to do clearly but the second 
	paragraph belongs to the beginning of introduction. 1.2 - state what the
	overarching problem of the thesis is but does not further clarify other
	challenges. 1.3 is fine overall but in the last paragraph you claim that
	reflection intensity is not needed to create and test object recognition
	algorithms. This seems very unlikely and you cite no sources supporting this
	claim. Reflection intensity is not only used for determining things such as
	the type of material but is also essential when determining the shape of an
	object, regardless of target geometry. This is because the intensity gives a
	lot of information about how to interpret the validity of the positional
	information of a certain point. This is used as a weight when calculating
	the final geometry, making it much more accurate. It is also used to
	minimize noise in the data by rejecting bad points entirely. The fact that
	some type of data is not emulated/created is reasonable for your short
	timeframe but this part should be rephrased. For example "Not producing this
	data gives a rougher picture or simpler data set, but shortens our 
	development time". Section 4.1.4 does not confirm that this data is good 
	enough, since no real comparison using an algorithm that takes reflection 
	intensity into account has been shown. 

\section*{Technical Background}

	Here the terms introduced in the previous chapter are further clarified as
	expected. 2.2.1 and 2.2.2 could maybe be merged to one section and 2.2
	could arguably be called Unity instead of Game Engine since Unity is the
	sole focus of the section. There could be a better picture in figure 2.1 to
	explain the concept of collision and the subtext in figure 2.2 is
	redundant. 2.2.3 is clear overall. 2.2.4 could be renamed to Pathfinding
	since that is the specific subject being discussed. It is a bit more rough
	but the point gets across.

\section*{Methods}

	This chapter's sections describes the methodology, requirement and the
	final choice that was made in creating the components. It sounds a bit wide
	for a "methods" section. 3.1 could become clearer by showing the math of
	the ray-casting algorithm in detail. Also, from what we could find reading
	up on Unity a bit, multithreaded physics simulation has been avalable since
	Unity 5, does this not work for ray tracing? 3.2 is fine overall but while
	the thesis uses scene in general to refer to the part that is created by
	the user, "world", "environment" and "space" is also sometimes used and it
	can get a bit confusing. The first paragraph in 3.2.2 is just a reiteration
	of 3.1. In 3.2.3 you mention that there are alternative methods and that
	there where tests to confirm decisions. Expand on this. 3.2.4 Is too small
	and seems to be out of place.  It could maybe be a separate section in this
	chapter. 3.1 is good in general but in 3.3.1, you choose a hash map 
	containing linked lists as storage structure for the data points. Why was 
	this chosen instead of, for example, a dynamic array? 3.4 was quite tough
	to parse and might be helped by some pictures and relating the concepts
	discussed to your application in paticular.

\section*{Results}

	As with the methods chapter, the results contain a lot of information about
	the implementation of the simulator. If all this information was moved to
	the same place, it should be possible to reduce the amount of text without
	losing any information. Consider creating a Implementation/Application 
	chapter and moving all implementation information there.
	
	4.3.2 - Methods for collision detection are compared. The increased 
	performance should therefore affect the physics time bar only, yet total 
	time is improved even more. How is this possible? Describe what other 
	changes you made or give arguments for how this is a valid result.

\section*{Discussion}

	The beginning of 5.1 just mentions further content and needs to be fleshed 
	out	some more. It is still a bit vague in 5.1.1 as to how things have been
	verified. This is maybe something that could be expanded upon in Methods
	beforehand. The first paragraph in 5.1.2 is written as more of a result and
	could instead be a reference to the results. The rest of the section could
	maybe expand on whether there were any alternative ways to reduce the
	performance that weren't explored. 5.2.1, 5.3 and 5.4 are reiterations of
	previous points. In 5.3.2 you mention that JSON is not directly readable, 
	but it is, in any text editor. You also conclude that the simulation 
	depends on the hardware, but this is always true to some extent. It would 
	help if you were more specific, maybe figure out some required minimum 
	spec. 5.5 is an interesting read.

\section*{Conclusion}

	The conclusion has some minor nitpicks mentioned in the appendix but gives
	overall a good summation.

\end{document}
