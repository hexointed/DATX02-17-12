\chapter{Introduction} 
	
	\section{Background}
		During the early 90’s many types of real time rendering algorithms were 
		developed and used. This was possible because they all ran in software 
		on the main computer processor, and software is easily changed and 
		enhanced. Photorealistic rendering could be done using a method called 
		Ray Tracing, but this was very slow. Improvements were made to this method
		and another algorithm formed called sphere tracing.
		 Sphere Tracing,  proved useful for Ray Tracing 
		3D fractals. It was still slow on current hardware and could 
		thus not be done in real time. Better graphics rendering
		(fast real time graphics) was a key selling point among competing 
		computer systems, so to improve graphics rendering, the 
		computer industry moved towards dedicated hardware based graphics 
		rendering. These systems were almost exclusively using polygon-based 
		rendering, and most Ray Tracing algorithms can not be rendered this 
		way. Today, 3D graphics cards have become more programmable and the 
		last 5 years have seen a small resurgence in the use of Sphere Tracing
		rendering. Simple scenes can now be rendered, using sphere tracing,
		 in real time using a state of the art consumer 3D graphics card.
		  This has a whole new set of pros 
		and cons when compared to polygon rendering. However, while 
		the algorithm seems to have a lot of potential for future 3D graphics, 
		that potential might be unreachable on current GPUs 
		(Graphics Processing Unit) due 
		to the underlying architecture being designed for polygon rendering. 
		This project will examine ideas for what could be achieved if graphics 
		cards were optimized and built specifically for Sphere Tracing Graphics
		instead of today’s polygon based hardware.
		 
	
	\section{Project goals}

		The main goal for this project was to design and implement a basic GPU 
		architecture that is designed to be able to accommodate the Sphere 
		Tracing algorithm. In addition, another aim was to look at possible 
		optimizations, both algorithmic	and hardware based, for this algorithm.
		
	\section{Scope}
		
		The scope of the project was threefold. First to research the algorithm
		and the improvements that already exist, secondly design a rudimentary GPU and
		finally to improve upon the design based upon what we assessed and
		figured out from the research.

		Each of these parts could span an entire bachelors thesis on their own, so in this
		project we instead did them on a best-effort basis as a compromise.
		When designing the GPU decisions were made to prioritize implementing the parts
		unique to the task before implementing the parts that all GPUs eventually need such
		as an advanced memory and cache architecture etc.
