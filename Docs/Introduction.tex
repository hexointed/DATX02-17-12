\chapter{Introduction} 	
%	\section{Background} 

	During the late '80s and through the '90s many types of real time rendering
	algorithms were developed and tested \cite{Hart1989}. This was possible
	because they ran in software on the	CPU (Central Processing Unit), and
	could thus be easily altered and enhanced. Photo realistic rendering could
	be achieved using a method called Ray Tracing, but this is slow for real time
	graphics\cite{PeterShirleyMichaelAshikhmin2005}. One version of this algorithm is method called Sphere Tracing. This could be used to render 3D
	fractals\cite{Hart1989}, but it was	still too slow to render complex
	geometry in real time\cite{Hart1989}. Better graphics rendering (fast real
	time graphics) was a key selling point among competing computer systems, so
	to improve rendering speed, the computer industry moved towards dedicated
	hardware based graphics rendering\cite{Houston2010}. These systems were
	almost exclusively using polygon-based	rendering, which is unsuitable for most
	Sphere Tracing algorithms. Today, 3D graphics cards have become more
	programmable and the last 10 years have seen a small resurgence in the use
	of Sphere Tracing rendering\cite{InigoQuilez}. Simple scenes can now be
	rendered using Sphere Tracing in real time using state-of-the-art consumer
	3D graphics cards. However, the full potential of the algorithm is not
	achievable on current GPUs due to the underlying architecture being
	designed for polygon rendering\cite{Houston2010}. This project examines the
	performance	potentials of graphics cards based on the Sphere Tracing 
	algorithm.

	\section{Note on shaders} 
		
		Throughout the report we will use the word \emph{shader} a lot. A
		shader is a historical term that today in modern industry terms refer to
		the code running on the GPU as opposed to on the CPU. It is called a
		\emph{shader} because initially only the color and illumination 
		attributes were	programmable, where as today's GPUs are vastly more 
		flexible. 

	\section{Project goals}
	
		The main goal for this project is to design and implement a basic GPU
		architecture that is designed to efficiently execute the Sphere Tracing
		algorithm. In striving towards efficiency, we examine Sphere Tracing in
		order to find possible optimizations, both algorithmic and hardware 
		based. For testing some of the optimizations performance, a software 
		shader is to be designed.
		
	\section{Scope}


		Writing a GPU for the first time, even a smaller one, and doing it well
		is a large undertaking for a single bachelors project. The project is
		therefore instead done on a best-effort basis where we try to implement
		the novel parts of the design, while leaving out generally well
		researched systems such as cache and memory management. When creating 
                software shaders, very simple and smooth running graphic scenes has been 
                prioritized over very complex scenes. This to have more resources for 
                optimizations and testing.



