\chapter{State of the Art}

Sphere tracing is currently being used to visualize complex data such as 
fractals, but there is no hardware designed to run it efficiently.Today's top 
of the line consumer 3d graphics cards are made for real time rendering of 
polygons. They achieve high speeds by executing the pixel color calculation in 
parallel. To make this efficient they use what is known as lockstepping, which 
means to have a group of calculation units always operate on the same 
instruction, but with differing input data. Doing this enables all of the 
cooperating cores to use only a single instruction memory and bus. This saves 
silicon real estate, enabling more cores per chip and thus greater performance. 
Pixels rendered by a polygon based method can usually be queued for calculation 
in such a way that pixels belonging to the same object in the scene can be 
grouped for computation. The drawback of the cooperating cores only being able 
to work on the same instruction at any one give time is thus greatly reduced. 
Using this hardware for Sphere Tracing (as is currently being done) will make 
this drawback significantly more burdening. Partly because pixels are not 
grouped by what object in the scene they belong to, and partly because even 
those that do belong to the same object will differ much more in their 
instruction execution path.




hårdvaruutveeckling idag
clash->verilog
