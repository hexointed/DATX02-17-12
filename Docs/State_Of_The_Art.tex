\chapter{State of the Art}

	\section{ Real Time Graphics Rendering on Current Hardware } 

		Sphere tracing is currently being used to visualize complex data such as
		fractals \cite{granskog2017}, but there is no hardware designed to run
		it efficiently. Today's top of the line consumer 3D graphics cards are
		instead made for real time rendering of polygons\cite{Houston2010}. They
		achieve good performance by executing rasterization and pixel color
		calculation in parallel between pixels. To make this efficient they use
		what is known as lockstepping, which means a group of calculation units
		that are always operating on the same instruction, but with differing
		input data. This enables all of the cooperating cores to use only a
		single instruction memory and bus. This reduces the area usage of the
		design, making it possible to add more cores per chip and thus achieve
		greater performance. Pixels rendered by polygon based methods can
		usually be queued for calculation in such a way that pixels belonging to
		the same object in the scene can be grouped for computation. This is
		possible because the GPU gets its graphics drawing instructions as a
		list of polygons, and consecutive polygons in this list mostly belong to
		the same 3D object. The drawback of a group of lockstepped cores; that
		all of the cores have to execute the same instruction at any given time,
		is greatly reduced since the color calculation for adjacent pixels of a
		certain 3D object will have the same set of instructions.
		
		Using this hardware for Sphere Tracing, as is currently being done,
		increases the overhead caused by this lockstepped design significantly.
		Order of pixel rendering in a polygon renderer is done by object and
		then by constituent polygons. Pixels can not be grouped by what object
		in the scene they belong to as easily in a Sphere Tracer, because Sphere
		Tracers do not have polygons and pixel order is usually based on the
		output image pixel order. Even if more grouping was introduced to a
		Sphere Tracer, the pixels in these groups differ much more in their
		instruction execution path, since the algorithm contains a loop that
		often varies greatly in the number of steps until completion. As
		described above, when the current instruction differs across cores in a
		group, some cores can not execute and have to wait until the remaining
		cores arrive at the instruction that they are wanting to execute.
		
	\section{ Hobbyists and Academia }

		Despite this, computer art hobbyists are using Sphere Tracing to
		produce quite stunning real time visuals on consumer PC's. They
		showcase the possibilities of the algorithm by reducing scene
		complexity and instead rendering using techniques that are commonly
		found in non real time Ray Tracers. Examples of such are true
		reflections and refractions, spacial repetition, object morphing and 3D
		fractals\cite{InigoQuilez}.  Inspired by research papers such as John
		C. Hart's 1996 paper\cite{Hart1996}, their success encouraged some
		to do academic research of their own, which led to new papers being
		written. A good example of this is the 2014 paper "Enhanced Sphere
		Tracing"\cite{Korndorfer2014}.

	\section{ Industry }		

		There are as of today no big commercial applications using Sphere Tracing
		that we know of, but Ray Tracing algorithms have long been in use in
		multiple computer graphics domains. For instance in film making
		\cite{Christensen2006}, where a comparatively large amount of time to
		render a scene can be acceptable since it can be computed ahead of time
		and saved as individual images, rather than producing the images in real
		time, which is needed for interactive visuals. Thus the computer
		graphics in movies can be much more photorealistic than interactively
		rendered computer graphics. An example of this would be the Ray Tracing
		engine RenderMan developed by Disney Pixar which they use in their movie 
		production \cite{Christensen2006}.

	\section{ Hardware Design Methods } 
	
		The design of Integrated Circuitry in the hardware industry has since
		the early '90s primarily been done in hardware description languages
		(HDLs) \cite{Chen2012}, where one describes the operation of a chip in a
		style similar to regular imperative programming languages. This
		descriptive code can then be compiled into a list of components and
		connections that constitutes the blueprint for that specific circuit.
		The most prevalent of these languages are VHDL and Verilog\cite{Chen2012}.
		While being a powerful aid in circuit design, compared to a more 
		layout-centric workflow, %  - fixaref
		these languages can be quite cumbersome to work with. They are verbose
		and require a fair amount of boilerplate code. This makes it more
		difficult to understand, follow, and also write code that performs
		complex tasks, since it can be more difficult to see the greater
		patterns in interconnecting code\cite{TODO}.
		
		This has resulted in Functional HDLs (FHDLs): ``Functional hardware
		description languages are a class of hardware description languages
		that emphasize on the ability to express higher level structural
		properties, such a parameterization and regularity. Due to such
		features as higher-order functions and polymorphism, parameterization
		in functional hardware description languages is more natural than the
		parameterization support found in the more traditional hardware
		description languages, like VHDL and Verilog'' \cite{Baaij2009}
		
		\label{FHDLs}
		HDLs have been around since the late '70s \cite{Chen2012}, but in recent
		times they have become more mature\cite{TODO}. Two notable modern FHDLs
		are \emph{Lava} and \clash \cite{Baaij2009, Bjesse1998}, which are both
		implemented in Haskell. This allows the same high-level interactive
		simulation of the program that normal Haskell programs enjoy. This means
		that instead of simulating the underlying circuit directly which is more
		time consuming, the design can be tested repeatedly at a faster pace,
		allowing faster development. This high level simulation can also be done
		in an interpreter enabling easy and rapid testing of code. These are
		called read-eval-print-loop interpreters.
