\chapter{State of the Art}

\section{ Real Time Graphics Rendering on Current Hardware } 

Sphere tracing is currently being used to visualize complex data such as 
fractals, but there is no hardware designed to run it efficiently. Today's top 
of the line consumer 3D graphics cards are made for real time rendering of 
polygons. They achieve high speeds by executing the pixel color calculation in 
parallel. To make this efficient they use what is known as lockstepping, which 
means a group of calculation units that is always operating on the same 
instruction, but with differing input data. Doing this enables all of the 
cooperating cores to use only a single instruction memory and bus. This saves 
silicon real estate, enabling more cores per chip and thus greater performance. 
Pixels rendered by a polygon based method can usually be queued for calculation 
in such a way that pixels belonging to the same object in the scene can be 
grouped for computation. The drawback of the cooperating cores only being able 
to work on the same instruction at any one given time is thus greatly reduced.

Using this hardware for Sphere Tracing, as is currently being done, will make 
this drawback significantly more burdening. Pixels are can not be grouped by
what object in the scene they belong to as easily in a Sphere Tracer, because
the order of pixel rendering in a polygon renderer is done by object and then
by its constituent polygons. Sphere Tracers do not have polygons and pixel 
order is usually based on the output image pixel order. Even if more grouping 
was introduced to a Sphere Tracer, the pixels in these groups differ much more 
in their instruction execution path, since the algorithm is more iterative and 
branching.

\section{ Hardware Design Methods  } 

The design of Integrated Circuitry in the industry has since the early '90s 
primarily been done in hardware description languages (HDL) \cite{ChenG}, 
where one describes the operation of a chip in a style similar to regular 
imperative programming languages. This descriptive code can then be compiled 
into a list of components and connections that constitutes the blueprint for 
building a circuit. The most prevalent of these languages are VHDL and 
Verilog. While being a great help to designers, compared to more manual 
design, these languages are quite cumbersome to work with. They are verbose, 
restrictively strict and require a fair amount of boilerplate code. Also, 
they are aimed at describing hardware at a low level, rather than the 
functions/calculations/tasks the resulting silicon is meant to perform. This 
makes it more difficult to understand, follow, and also write code that 
performs complex tasks and advanced signal processing.

Enter Functional HDLs (FHDL): 
"Functional hardware description languages are a class of hardware description languages that emphasize on the ability to express higher level structural properties, such a parameterization and regularity.  Due to such features as higher-order functions and polymorphism, parameterization in functional hardware description languages is more natural than the parameterization support found in the more traditional hardware description languages, like VHDL and Verilog" \cite{Baaij}

Functional HDLs have been around since the late '70s \cite{ChenG}, but in 
recent times they have become more mature. In particular the Clash language 
caught our eye, by taking the elegance of FHDLs to the next level by lifting 
circuit state to the type level.
