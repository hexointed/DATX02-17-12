\chapter{State of the Art}

	\section{ Real Time Graphics Rendering on Current Hardware } 

		Sphere tracing is currently being used to visualize complex data such
		as fractals \cite{TODO}, but there is no hardware designed to run it
		efficiently.  Today's top of the line consumer 3D graphics cards are
		made for real time rendering of polygons. They achieve good performance
		by executing rasterization and pixel color calculation in parallel
		between pixels. To make this efficient they use what is known as
		lockstepping, which means a group of calculation units that are always
		operating on the same instruction, but with differing input data. This
		enables all of the cooperating cores to use only a single instruction
		memory and bus. This reduces the area usage of the design, making it
		possible to add more cores per chip and thus achieve greater
		performance. Pixels rendered by polygon based methods can usually be
		queued for calculation in such a way that pixels belonging to the same
		object in the scene can be grouped for computation. The drawback of
		lock-stepped cores only being able to work on the same instruction at a
		time is thus greatly reduced.
		
		Using this hardware for Sphere Tracing, as is currently being done,
		increases the overhead caused by this lock-stepped design
		significantly. Pixels can not be grouped by what object in the scene
		they belong to as easily in a Sphere Tracer, because the order of pixel
		rendering in a polygon renderer is done by object and then by its
		constituent polygons. Sphere Tracers do not have polygons and pixel
		order is usually based on the output image pixel order. Even if more
		grouping was introduced to a Sphere Tracer, the pixels in these groups
		differ much more in their instruction execution path, since the
		algorithm is more iterative and often varies greatly in number of steps
		until completion.
		
		\section{ Hobbyists and Academia }
		
		Despite this, computer art hobbyists are using Sphere Tracing to produce
		some quite stunning real time visuals on consumer PC's. They showcase 
		the possibilities of the algorithm by reducing scene complexity and 
		instead rendering using techniques that are commonly only found in non 
		real time Ray Tracers. Examples of such are true reflections and 
		refractions, spacial repetition, object morphing and 3D fractals. 
		For some examples of this check out ???  \cite{InigoQuilez}.
		Early on, they were inspired by research papers such as John C. Hart's 
		1996 paper\cite{Hart1996}. Their success encouraged some of them to do 
		academic research of their own, which led to new papers being written. A 
		good example of this is the 2014 paper "Enhanced Sphere 
		Tracing"\cite{Korndorfer2014}.

		\section{ Industry }		
		% Kanske helt orelevant ?
		There are as of today no big commercial applications using Sphere
		Tracing that we know of, but Ray Tracing algorithms have long been in 
		use in multiple computer graphics domains. For instance in film making 
		\cite{TODO}, where a comparatively large amount of time to render a 
		scene can be acceptable but the demand for realism is higher than in 
		real time graphics. An example of this would be the Ray Tracing engine 
		RenderMan developed by Disney Pixar which is used in their movies \cite{TODO}.
	
	\section{ Hardware Design Methods } 
	
		The design of Integrated Circuitry in the industry has since the early
		'90s primarily been done in hardware description languages (HDL)
		\cite{Chen2012}, where one describes the operation of a chip in a style
		similar to regular imperative programming languages. This descriptive
		code can then be compiled into a list of components and connections
		that constitutes the blueprint for building a circuit. The most
		prevalent of these languages are VHDL and Verilog\cite{TODO}. While
		being a great help to designers, compared to more manual design, these
		languages can be quite cumbersome to work with. They are verbose and 
		require a fair amount of boilerplate code. This makes it more difficult
		to understand, follow, and also write code that performs complex tasks, 
		since it can be more difficult to see the greater patterns in 
		interconnecting code.
		
		This has resulted in Functional HDLs (FHDL): ``Functional hardware
		description languages are a class of hardware description languages
		that emphasize on the ability to express higher level structural
		properties, such a parameterization and regularity. Due to such
		features as higher-order functions and polymorphism, parameterization
		in functional hardware description languages is more natural than the
		parameterization support found in the more traditional hardware
		description languages, like VHDL and Verilog'' \cite{Baaij2009}
		
		Functional HDLs have been around since the late '70s \cite{Chen2012},
		but in recent times they have become more mature. There are in
		particular two FHDLs, \emph{Lava} and \clash \cite{Baaij2009,
		Bjesse1998}, that are implemented in Haskell.  This allows the same
		interactive type checking and high-level simulation of the program that
		normal Haskell programs enjoy. This means that instead of simulating
		the underlying circuit directly which is more time consuming, the
		design can be tested and iterated over at a faster pace,
		allowing for faster development.
