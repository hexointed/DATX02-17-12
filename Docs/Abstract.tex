\thispagestyle{plain}			% Supress header 
\setlength{\parskip}{10pt}
\setlength{\parindent}{0pt}

\section*{Abstract}

	This thesis documents the result of a graphics processing unit (GPU) that
	we have designed to be able to run a specific rendering algorithm called
	Sphere Tracing, a sub algorithm of Ray Tracing. The GPU is written in the functional
	hardware design programming language \clash. A software shader were implemented 
	in GLSL to be able to research the algorithm.
	
	Real-time rendering performance has always been a significant factor for 3D
	graphic cards. However, 3D graphic cards today are optimized and built for
	polygon-based rendering. Therefore, Sphere Tracing rendering is slower on
	this hardware. Which leads to the question, how fast can the Sphere Tracing 
	algorithm run on hardware designed for this algorithm. The aim of this thesis 
	is to implement and design a basic GPU (Graphics processing unit) that is 
	designed for Sphere Tracing, using \clash,
	a functional hardware description language. We assessed the potential
	performance of the Sphere Tracing algorithm using our software shader.
	
	 
	This led to hardware and software optimizations, Orthogonal Culling, hardware square 
	roots, Bounding Spheres, Ray Bunching, of which some where implemented. 
	The results indicate that it is possible to increase the performance of the 
	Sphere Tracing algorithm. In addition possible future work, software 
	and hardware improvements, is discussed.

	% KEYWORDS (MAXIMUM 10 WORDS)
	\vfill
	Keywords: sphere tracing, ray marching, ray tracing, GPU, real-time rendering.

\newpage
\thispagestyle{plain}

\section*{Sammanfattning}
	
	Denna uppsats dokumenterar resultaten kring en grafikenhet (GPU) som vi
	designat med avseende att köra en specifik renderingsalgoritm kallad Sphere
	Tracing, en typ av Ray Tracing. GPUn är skriven i det funktionella
	hårdvarubeskrivande programmeringsspråket \clash. För att undersöka algoritmen 
	skapade vi en mjukvarushader i GLSL.
	
	Hastigheten på realtidsrenderingen hos 3D-grafikkort har alltid varit ett
	viktigt säljargument. Dock, när det kommer till Sphere Tracing-rendering
	jämfört med polygonbaserade renderingsmetoder så har den förstnämda
	historiskt sett varit långsammare. Detta har lett till att man
	vidareutvecklat hårdvaran specifikt för att accelerera polygonrendering
	vilket leder till frågan hur pass snabbt Sphere Tracing kan köras om liknande
	hårdvara skräddarsys för just den algoritmen. Avsikten med denna rapport är
	att ge en idé om vilka prestandaökningar som finns att uppnå och även implementera ett
	alternativt realtidsrenderande 3D-grafikkort i \clash. För att studera potentiala 
	förbättringar av algoritmen användes shadern.
	
	Detta bidrog till hårdvaru och mjukvaru förbättringar, Orthogonal Culling, 
	kvadratrötter i hårdvara, Bounding Spheres, Ray Bounching. Vissa av dessa blev 
	förverkligade i projektet. Resultaten pekar på att det går att öka prestandan 
	för algoritmen. Utöver detta diskuteras även mjukvaru och hårdvaru 
	förbättringar för framtida arbete.

	% KEYWORDS (MAXIMUM 10 WORDS)
	\vfill
	Nyckelord: sphere tracing, ray marching, ray tracing, realtidsrendering.


\newpage
\thispagestyle{empty}
\mbox{}
