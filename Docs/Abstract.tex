\thispagestyle{plain}			% Supress header 
\setlength{\parskip}{0pt plus 1.0pt}

\section*{Abstract}
	
	Real-time rendering speed has always been a significant factor for 3D
	graphic cards. However, 3D graphic cards today are optimized and built for
	the polygon-based rendering. Therefore, Sphere Tracing rendering is slower
	in this hardware.  The aim of this thesis is to implement and design a
	basic GPU (Graphic proccesing unit) that could execute Sphere tracing.
	Subsequent to this, some optimizations were implemented in order to further
	assess the potential performance of the Sphere Tracing algorithm.
	
	This thesis documents the result of us examining the Sphere Tracing
	algorithm and designing a Graphics Processing Unit (GPU) to run it, using
	\clash, a functional hardware description language. In addition, possible
	future work are discussed.

	% KEYWORDS (MAXIMUM 10 WORDS)
	\vfill
	Keywords: sphere tracing, ray marching, ray tracing, GPU, real-time rendering.

\newpage
\thispagestyle{plain}

\section*{Sammanfattning}
	
	Denna uppsats dokumenterar resultaten kring en grafikenhet (GPU) som vi
	designat med avseende att köra en specifik rendreringsalgoritm kallad Sphere
	Tracing, en typ av Ray Tracing. GPUn är skriven i det funktionella
	hårdvarubeskrivande programmeringsspråket (FHDL) \clash.
	
	Hastigheten på realtidsrenderingen hos 3D grafikkort har alltid varit en
	viktig säljpunkt. Dock, när det kommer till Sphere Tracing rendrering jämfört
	med polygonbaserade rendreringsmetoder så har den förstnämda historiskt sett
	varit långsammare. Detta har lett till att man vidareutvecklat hårdvaran
	specifikt för att accelerera polygonrendrering vilket leder till frågan hur
	pass snabbt Sphere Tracing kan köras om liknande hårdvara skräddarsys för
	just den algoritmen. Avsikten med denna rapporten är att ge en idé om vilka
	prestandaökningar som finns att uppnå med ett alternativt realtidsrenderande
	3D grafikkort.
	
	% KEYWORDS (MAXIMUM 10 WORDS)
	\vfill
	Nyckelord: bollkoll, sphere tracing, ray marching, ray tracing, realtidsrendering.


\newpage
\thispagestyle{empty}
\mbox{}
