\thispagestyle{plain}			% Supress header 
\setlength{\parskip}{10pt}
\setlength{\parindent}{0pt}

\section*{Abstract}



	This thesis documents our experiences and conclusions during the design of  
	a graphics processing unit (GPU), specifically designed and optimized for 
	executing the Sphere Tracing algorithm. This is a variation of the standard Ray Tracing algorithm. The GPU is written 
	in the functional hardware design programming language \clash. A shader was 
	implemented in GLSL to enable algorithm research.
	
	Real-time rendering performance has always been a significant factor for 3D
	graphic cards. However, using the Sphere Tracing algorithm on conventional 3D graphics card is slow because they're designed and built for polygon based rendering. Which leads to the question, how fast can the Sphere Tracing 
	algorithm run on hardware designed for this algorithm. We assessed 
	potential performance improvements of the Sphere Tracing algorithm using 
	our software shader. We designed hardware and software optimizations, and 
	implemented some in \clash and our shaders. The results indicate that it is 
	possible to increase the performance of the Sphere Tracing algorithm. In 
	addition, future work in hardware and software improvements is discussed.

	% KEYWORDS (MAXIMUM 10 WORDS)
	\vfill
	Keywords: sphere tracing, ray marching, ray tracing, GPU, real-time rendering.

\newpage
\thispagestyle{plain}

\section*{Sammanfattning}
	
	Denna uppsats dokumenterar våra upplevelser och slutsatser under 
	utvecklingen av en grafikprocessor (GPU) som vi	designat och optimerat 
	för att exekvera renderingsalgoritmen Sphere Tracing, en typ av Ray 
	Tracing. GPU:n är skriven i det funktionella hårdvarubeskrivningsspråket 
	\clash. För att undersöka algoritmen skapade vi en shader i GLSL.
	
	Hastigheten på realtidsrenderingen hos 3D-grafikkort har alltid varit ett
	viktigt säljargument. Vanliga 3D-grafikkort är designade för att rendera polygonbaserad grafik så Sphere Tracing algoritmen uppnår inte sin potentiella prestanda. Detta leder till frågan hur pass snabbt Sphere Tracing 
	egentligen skulle kunna köras om liknande hårdvara skräddarsys för just 
	denna algoritm. Avsikten med rapporten är dels att ge en idé om vilka 
	prestandaökningar som finns att uppnå, och dels att implementera ett
	alternativt realtidsrenderande 3D-grafikkort i \clash. 
	
	Detta bidrog till hårdvaru- och mjukvaruförbättringar, varav vissa 
	implementerades i \clash och våra shaders. Resultaten pekar på att det går 
	att öka prestandan för algoritmen. Utöver detta diskuteras även mjukvaru- 
	och hårdvaruförbättringar för framtida projekt.

	% KEYWORDS (MAXIMUM 10 WORDS)
	\vfill
	Nyckelord: sphere tracing, ray marching, ray tracing, realtidsrendering.


\newpage
\thispagestyle{empty}
\mbox{}
