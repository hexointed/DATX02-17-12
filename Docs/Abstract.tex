\thispagestyle{plain}			% Supress header 
\setlength{\parskip}{0pt plus 1.0pt}
\section*{Abstract}
	
	This thesis documents the result of using the Sphere Tracing algorithm and
	designing a Graphics Processing Unit (GPU) to run it, using Clash, a
	functional hardware description language. Real-time rendering speed has
	always been a significant factor for 3D graphic cards. However, Sphere
	Tracing rendering is slower in comparison with polygon-based rendering.
	Therefore, the 3D graphic cards today are optimized and built for the
	latter. Although Sphere Tracing is slower there are still possibly some
	performance potentials that have not yet been reached, which could improve
	the performance of the algorithm.
	
	The thesis begins with a description of the Sphere Tracing algorithm, then
	continues with descriptions of the implementation, optimizations and results
	of the hardware and software programs.
	
	The purpose of this report is to give a hint of what, if anything, could be
	gained from an alternative approach to real time rendering 3D graphics
	cards.

	% KEYWORDS (MAXIMUM 10 WORDS)
	\vfill
	Keywords: ray-marching, GPU, real-time rendering.

\newpage
\thispagestyle{plain}

\section*{Sammanfattning}
	
	Sammanfattning på svenska.
	
	% KEYWORDS (MAXIMUM 10 WORDS)
	\vfill
	Nyckelord: ray-marching, GPU, realtidsrendering.


\newpage
\thispagestyle{empty}
\mbox{}
