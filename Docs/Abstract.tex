\thispagestyle{plain}			% Supress header 
\setlength{\parskip}{10pt}
\setlength{\parindent}{0pt}

\section*{Abstract}



	This thesis documents our experiences and conclusions during the design of a
	graphics processing unit (GPU), specifically designed and optimized for
	executing the Sphere Tracing algorithm, which is a variation of the standard
	Ray Tracing algorithm. The GPU is written in the functional hardware description
	language \clash. A shader was also implemented in GLSL to enable
	algorithm research.
	
	Real-time rendering performance has always been a significant factor for 3D
	graphics cards. However, using the Sphere Tracing algorithm on conventional
	3D graphics cards is slow because they are designed and built for polygon
	based rendering. We assessed potential performance improvements using our
	GLSL shader. The results indicate that it is possible to increase the
	performance of the Sphere Tracing algorithm.

% KEYWORDS (MAXIMUM 10 WORDS)
	\vfill Keywords: sphere tracing, ray marching, ray tracing, GPU, real-time
	rendering.

\newpage
\thispagestyle{plain}

\section*{Sammanfattning}
	
	Denna uppsats dokumenterar våra upplevelser och slutsatser under 
	utvecklingen av en grafikprocessor (GPU) som vi	designat och optimerat 
	för att exekvera renderingsalgoritmen Sphere Tracing, en typ av Ray 
	Tracing. GPU:n är skriven i det funktionella hårdvarubeskrivningsspråket 
	\clash. För att möjliggöra algoritmstudier implementerades en shader i GLSL.
	
	Hastigheten på realtidsrenderingen hos 3D-grafikkort har alltid varit en
	viktigt faktor. Dock, är det långsamt att använda Sphere Tracing algoritmen
	för att rendera grafik på konventionella grafikkort eftersom de är designade
	och byggda för polygonbaserad rendering. Vi undersökte potentiella
	prestandaförbättringar för Sphere Tracing algoritmen i vår GLSL shader.
	Resultaten indikerar att det är möjligt att förbättra prestandan för Sphere
	Tracing algoritmen.

	% KEYWORDS (MAXIMUM 10 WORDS)
	\vfill
	Nyckelord: sphere tracing, ray marching, ray tracing, realtidsrendering.


\newpage
\thispagestyle{empty}
\mbox{}
