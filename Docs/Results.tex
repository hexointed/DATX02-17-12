\chapter{Results}

	Here are different improvements of the GPU that's been implemented and tested.
	Also hypothetical improvements that has only been discussed and not
	implemented.

	\section{Software Shader Performance}

		The shader performed as expected, a conventional graphics card is 
		capable of rendering scenes with low scene complexity in real-time
		using sphere tracing.

		However scenes can easily be made to look a lot more complex than they 
		are, for example by using mod fields or fractals. On the hardware that 
		the shader was tested on (Geforce GTX 1060M) 20 reflective sphere can
		be rendered in real time in fullHD using our performance enhancing 
		algorithm.


	\section{GPU Performance}

		It is important to note that almost no effort was spent trying to
		optimize the performance of our design. 

	\section{Optimizations}
		
		During this project, a number of optimizations were discussed and 
		developed, the developed are explained here and the theoretical ones
		are brought up in the discussion section. Some of these are based on earlier
		work, with some others we believe are quite different from optimizations
		that have been discussed for sphere tracing previously.

		\subsection{Orthogonal culling}
			Orthogonal culling was implemented in the software shader and increased 
			the performance.

			The tests was done in 3840x2160 on a GTX 1060M, which is a mid-high range 
			GPU.

			The test was performed such that we put an increasing number of solid-colored spheres in a 
			plane in front of the camera. Because of this, the spheres are not obstructed
			by other spheres, making this the best possible scenario for the optimization.

			with optimization:

			\begin{tabular}{lll}
				objects: & fps:	\\ \hline 
				1	& 600 & 350	\\ 
				5	& 430 & 180	\\			
				10	& 290 & 98	\\
				15	& 85 & 13	\\
				20	& 58 & 9	\\
				25	& 40 & 6	\\
				30	& 29 & 4	\\
				35	& 7 & 3		\\
				40	& 6 & 2		\\
				45	& 4	& 1.5	\\
			\end{tabular}

			%grafer och lite text


		
		
		

