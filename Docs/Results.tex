\chapter{Results}

	Here different improvements of the GPU that's been implemented and tested.
	Also hypothetical improvements that's only been discussed and not
	implemented due to lack of time.  

	\section{Ray grouping}
	
		Ray grouping is an optimization that lowers the number of computations
		the GPU have to perform. And by lowering the number of computations it
		increases the speed of the GPU.
		
		It works like the name implicates by grouping adjacent rays together.
		If a specific pixel N is to be rendered we group the adjacent pixels
		into a "combined pixel". Then we march along the ray of N. If any of
		the other pixels are not inside the minimum distance spheres (MDS) the
		combined pixel is split up into two smaller groups. Each of the groups
		then repeat the first step again individually along their new center
		pixel's ray. This is repeated a number of times depending on the scene.
		The closer to a target it gets, more subgroups will be created due to
		the MDS's volumes will decrease.
		
		This optimization was implemented and we achieved an increase in
		performance as excepted. The optimization decreases the number of
		computations made by the formula ? below.  Sum o$f_1^s(1/N_s)$ where N
		= number of pixels in the group and s = number of groups
	
		%#picture of ray grouping
	
	\section{Cull}
	
	\section{Bounding spheres}

		Bounding spheres technique is somewhat similar to the normal sphere
		tracing. The difference is that when you march you march to the edge of
		the MDS and then you march to the next MDS edge. With bounding sphere
		technique you march a small distance further outside the MDS edge. You
		then compare the original MDS with the new MDS if these two spheres
		overlap in any way we can march that little bit further. By marching
		that little bit further, decrease in the number of times marched is
		achieved. Giving an increase in performance.  \\ This optimization
		didn't get implemented due to lack of time. But we do believe this will
		work and have a positive effect on performance. this will be discussed
		more in chapter 6.
	
		%picture of example bounding sphere
