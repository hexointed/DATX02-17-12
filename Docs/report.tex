\documentclass[a4paper]{article}

%Figure support for \includegraphics
\usepackage{graphicx}
%Code listings
\usepackage{listings}
\usepackage{float}
\usepackage{verbatim}

%Math fonts
\usepackage{amsfonts}
\usepackage{amsmath}

%Control of caption font for figures and tables
\usepackage[font=small]{caption}

%Swedish is supported once we add these packages
\usepackage[swedish]{babel}
\usepackage[T1]{fontenc}
\usepackage[utf8]{inputenc}

%short footnote macro
\usepackage{scrextend}
\newcommand{\pfoot}[2]{\footnote{#2\label{#1}}}
\newcommand{\fref}[1]{\footref{#1}}

%short underscore
\newcommand{\mstr}{\textunderscore}

\author{Elias Forsberg, Jon Johnsson, Chi-Thong Luong, 
	Björn Strömberg, Jesper Åberg}
\title{A ray marching GPU}

\begin{document}

\maketitle

%----------------------------------------------------------------------
\chapter{Introduction}

The aim of this project is to develop the fundamental building blocks for a real-time 3D 
graphics processing unit (GPU) that is optimized to run a rendering algorithm which is 
different to the Polygon-based rendering that is almost exclusively used in today's 3D 
graphics cards. During the early 90’s many types of rendering algorithms were developed and 
used, since they all ran in software on the main computer processor. Better performance 
(better graphics) was a key selling point among competing computer systems, the computer 
industry moved towards dedicated hardware based rendering systems. These systems were 
almost exclusively using polygon-based rendering. Today, 3D graphics cards have become more 
programmable but are still based on the old paradigm of polygon rendering. The last 5 years 
have seen a small resurgence in the use of a sphere tracing rendering algorithm known as Ray-
Marching. This has a whole new set of pros and cons when compared to the polygon renderers. 
However, while the algorithm seems to have a lot of potential for future 3D graphics, that 
potential might be unreachable on current GPUs due to the underlying architecture being 
designed with other things in mind. This project will examine what could be achieved if 
graphics cards were optimized for ray-marching instead of drawing polygons.

\newpage
\appendix

\end{document}
